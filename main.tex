\documentclass[conference]{IEEEtran}
\usepackage{savesym}
\usepackage{amsmath,amssymb,amsfonts,amsthm}
\usepackage[numbers]{natbib}
\usepackage{algorithm}
\usepackage[noend]{algpseudocode}
\usepackage{comment}
\usepackage{xspace}
\usepackage{scalerel}
\usepackage{xcolor}
\usepackage[caption=false,font=footnotesize]{subfig}
\usepackage[left]{showlabels}
\usepackage{multicol}
\usepackage[bookmarks=true]{hyperref}

\begin{document}

\title{	Online, interactive user guidance for \\
		high-dimensional, constrained motion planning }

\author{Fahad Islam, Oren Salzman and Maxim Likhachev}


\maketitle
\thispagestyle{empty}
\pagestyle{empty}

\def\frechet{Fr\'echet\xspace}

\newcommand{\cupdot}{\mathbin{\mathaccent\cdot\cup}}

%unlabeled PSPACE-hardness paper
\newcommand{\mtm}{\emph{multi-to-multi}\xspace}
\newcommand{\mts}{\emph{multi-to-single}\xspace}
\newcommand{\sts}{\emph{multi-to-single-restricted}\xspace}
\newcommand{\dtd}{\emph{single-to-single}\xspace}

\newcommand{\cte}{\emph{full-to-edge}\xspace}
\newcommand{\ctc}{\emph{full-to-full}\xspace}
\newcommand{\ete}{\emph{edge-to-edge}\xspace}

\newcommand{\AND}{{\sc and}\xspace}
\newcommand{\OR}{{\sc or}\xspace}

%tex tools
\newcommand{\ignore}[1]{}

%%% algorithms
\def\vor{\text{Vor}}

\def\P{\mathcal{P}} \def\C{\mathcal{C}} \def\H{\mathcal{H}}
\def\F{\mathcal{F}} \def\U{\mathcal{U}} \def\L{\mathcal{L}}
\def\O{\mathcal{O}} \def\I{\mathcal{I}} \def\E{\mathcal{E}}
\def\S{\mathcal{S}} \def\G{\mathcal{G}} \def\Q{\mathcal{Q}}
\def\I{\mathcal{I}} \def\T{\mathcal{T}} \def\L{\mathcal{L}}
\def\N{\mathcal{N}} \def\V{\mathcal{V}} \def\B{\mathcal{B}}
\def\D{\mathcal{D}} \def\W{\mathcal{W}} \def\R{\mathcal{R}}
\def\M{\mathcal{M}} \def\X{\mathcal{X}} \def\A{\mathcal{A}}
\def\Y{\mathcal{Y}} \def\L{\mathcal{L}}

\def\dS{\mathbb{S}} \def\dT{\mathbb{T}} \def\dC{\mathbb{C}}
\def\dG{\mathbb{G}} \def\dD{\mathbb{D}} \def\dV{\mathbb{V}}
\def\dH{\mathbb{H}} \def\dN{\mathbb{N}} \def\dE{\mathbb{E}}
\def\dR{\mathbb{R}} \def\dM{\mathbb{M}} \def\dm{\mathbb{m}}
\def\dB{\mathbb{B}} \def\dI{\mathbb{I}} \def\dM{\mathbb{M}}

\def\eps{\varepsilon}
\def\obs{\mathrm{obs}}

\newcommand{\sbs}{sampling-based\xspace}
\newcommand{\mr}{multi-robot\xspace}
\newcommand{\mpl}{motion planning\xspace}
\newcommand{\cs}{configuration space\xspace}
\newcommand{\conf}{configuration\xspace}
\newcommand{\confs}{configurations\xspace}

% programming
\newcommand{\Cpp}{C\raise.08ex\hbox{\tt ++}\xspace}



\newcommand{\ch}{\mathrm{ch}}
\newcommand{\pspace}{{\sc pspace}\xspace}
\newcommand{\np}{{\sc np}\xspace}
\newcommand{\degree}{\ensuremath{^\circ}}
\newcommand{\argmin}{\operatornamewithlimits{argmin}}


\newcommand{\dist}{\textup{dist}}

\newcommand{\Cfree}{\C_{\textup{free}}}
\newcommand{\Cforb}{\C_{\textup{forb}}}

\newtheorem{lemma}{Lemma}
\newtheorem{theorem}{Theorem}
\newtheorem{corollary}{Corollary}
\newtheorem{claim}{Claim}

\theoremstyle{definition}
\newtheorem{definition}{Definition}
\newtheorem{remark}{Remark}
\theoremstyle{plain}
\newtheorem{observation}{Observation}

\def\os#1{\textcolor{blue}{#1}}


\makeatletter
\def\thmhead@plain#1#2#3{%
  \thmname{#1}\thmnumber{\@ifnotempty{#1}{ }\@upn{#2}}%
  \thmnote{ {\the\thm@notefont#3}}}
\let\thmhead\thmhead@plain
\makeatother

\def\todo#1{\textcolor{blue}{\textbf{TODO:} #1}}
\def\new#1{\textcolor{magenta}{#1}}
\def\old#1{\textcolor{red}{#1}}

\def\removed#1{\textcolor{green}{#1}}
%\def\removed#1{}
%%% Local Variables:
%%% mode: plain-tex
%%% TeX-master: "main"
%%% End:


\begin{abstract}

\end{abstract}

\IEEEpeerreviewmaketitle

%%%%%%%%%%%%%%%%%%%%%%%%%%%%%%%%%%%%%%%%%%%%%%%%%%%%%
%Intro
%%%%%%%%%%%%%%%%%%%%%%%%%%%%%%%%%%%%%%%%%%%%%%%%%%%%%
%\begin{comment}
\section{Introduction}
\label{sec:intro}

Motion-planning is a fundamental problem in robotics that has been studied for over four decades. 
However, efficiently planning paths in high-dimensional, constrained spaces remains an ongoing challenge.
One approach to address this challenge is to incorporate user guidance.
While there has been much work on planning using human demonstration (see, e.g.,~\cite{HS16, PHCL16, SHLA16, YA17} for a partial list), there has been surprisingly little research incorporating guidance as an interactive part of the planning loop.

Broadly speaking, interactive planning has been typically used in the context of sampling-based motion-planning algorithms~\cite{L06}.
User guidance is employed by biasing the sampling-scheme of the planner.
This can be done by identifying regions in the \emph{workspace} that should be avoided / explored~\cite{DSJA14, MTMKDC15, YPB15}.
Alternatively interactive devices such as a 3D mouse or a haptic arm have been used to generate paths in a (low-dimensional) configuration space. This path is then used by a planner to bias its sampling domain~\cite{BTFF16, FTF09, TFF12}.
Interestingly, in all the examples mentioned, an implicit assumption taken is that the user is dedicated to guide and interact with the planning algorithm.

We are interested in planning in high-dimensional, constrained spaces such as those encountered by a humanoid robot (see Fig.~\ref{Fig:robot}).
In such settings, workspace regions often give little guidance to the planner due to the dimension of the configuration space as well as the physical constraints of the robot.
Due to similar reasons, guiding the planner by suggesting complete paths in the configuration space, even at a slow rate, is extremely hard, even for expert users.

Thus,  while beneficial, user guidance is  time consuming and should be employed scarcely.
Our key insight is that carefully chosen individual configurations suggested by a user can be used to effectively guide the planner.
Transforming this insight into a planning framework requires addressing three fundamental questions:

\begin{itemize}
	\item[\textbf{Q1.}] When should the planner ask the user for guidance?
	\item[\textbf{Q2.}] What form should the user's guidance take?
	\item[\textbf{Q3.}] How should the guidance be used?
\end{itemize}
 
In the following, we detail how our planning framework addresses each of these questions.

%\algrenewcommand\algorithmicindent{.8em}
\begin{algorithm}[tb]
\caption{Planning framework}
\label{alg:main}	
\begin{algorithmic}[1]
%\small
\While{Solution not found} 
	\While{Not in local minima} 
		\State Run MHA*
	\EndWhile
	
	\State Get user guidance
	\State Add new heuristic and queue to MHA*
	\While{In local minima} 
		\State Run MHA*
		\Comment{with bias towards user's guide}
	\EndWhile
	\State Remove new queue from MHA*

\EndWhile
\end{algorithmic}
\end{algorithm}

\section{Planning framework}
\label{sec:planning}
In this work we build upon multi-heuristic A* (MHA*)~\cite{ASNHL16}, a search-based planning algorithm that takes in multiple, arbitrarily inadmissible heuristic functions in addition to a single consistent heuristic.
It uses all of them simultaneously to search in a principled, A*-like manner, that was shown to be both complete and ensures bounds on sub-optimality. 
This allows the search to efficiently combine the guiding powers of different heuristic functions. 

\os{refer to alg.~\ref{alg:main} here and within subsections}

\subsection{Invoking user guidance}
\label{sec:q1}
Invoking user guidance requires (i)~disrupting a user and (ii)~waiting a non-negligible time for the user to produce guidance.
Thus, we aim at invoking the user only when the planner is in a \emph{local minima}. Namely when it identifies that its search does not progress towards the goal. 

Search-based planning algorithms, such as MHA*, make use of heuristic functions to guide the search. These heuristic functions can be used to identify in a principled manner when the planner is in a local minima. 
Specifically, recall that in a search-based planning algorithm,  we proceed by iteratively choosing  the current-best state from a priority queue and computing all its successors. We suggest several approaches to identify when the planner is in a local minima.

The first is \os{describe expansion delay}.
The second is \os{describe heuristic progress}.
% when the key of the node popped at the current iteration is larger than the key of the node popped at the previous iteration.

Once a local minima is identified, it can then be used to trigger the event that the search does not progress towards the goal. 
However, should we immediately ask for user guidance? If we are in a shallow local minima, providing the planner with additional computation time (possibly much shorter than the time required by the user to produce guidance) may be sufficient.
This is an example of the ski rental problem in which there is a choice between continuing to pay a repeating cost (letting the algorithm try to escape its local minima) or paying a one-time cost which eliminates or reduces the repeating cost (invoking user guidance).
Indeed, by using a randomized approach one can obtain a $\frac{e}{e-1}\approx1.58$ competitive ratio\footnote{The competitive ratio of an online algorithm is the ratio between the performance of an online algorithm to the performance of an optimal offline algorithm.}~\cite{KMMO94}.
Interestingly, this algorithm has been previously used to identify when robots should ask for help from a user~\cite{RV12}.

\subsection{Form of user guidance}
Describe GUI

Configuration vs. sub-manifold


\subsection{Using user guidance}
Recall that MHA* has at least one admissible heuristic $h_{0}$.
Furthermore, we assume that there exists a family of admissible\footnote{\os{do they have to be admissible?}} heuristic function $\H$, such that for every configuration~$q$, there exists a heuristic $h_q \in \H$ where $h_q(s)$ estimates the cost from to reach $q$ from state $s$.
%%%%

Given a user guide in the form of a configuration $\hat{q}$, we dynamically generate a new heuristic $\hat{h} = h_{\hat{q}} + h_0$.
Namely,~$\hat{h}$ estimates the 
cost to reach the goal (via the term $h_0$) by passing through the guide $\hat{q}$ (via the term $h_{\hat{q}}$).
Equipped with the heuristic $\hat{h}$, we add a new queue to the MHA* algorithm prioritized using the heuristic $\hat{h}$. 
States expanded using this queue will be biased towards $\hat{q}$.
Note that in the MHA* algorithm,nodes are shared between the different queues.
Thus, once a state has been found that can be used to get the planner out of the local minima, it will be expanded by the other queues using their heuristics.
Once this is detected (using the same mechanism to detect the local minima detailed in Sec.~\ref{sec:q1}), the newly-added queue is removed.

%%%%%%%%%%%%%%%%%%%%%%%%%%%%%%%%%%%%%%%%%%%%%%%%%%%%%
%Evaluation
%%%%%%%%%%%%%%%%%%%%%%%%%%%%%%%%%%%%%%%%%%%%%%%%%%%%%
\section{Evaluation}\label{sec:evaluation}

\section{Discussion}\label{sec:discussion}
\os{discuss limitation, future work}

%\newpage

%\bibliographystyle{plainnat}
\bibliographystyle{abbrv}
\bibliography{bibliography}

\end{document}

%%% Local Variables:
%%% mode: latex
%%% TeX-master: t
%%% End:
